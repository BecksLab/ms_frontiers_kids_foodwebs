% Options for packages loaded elsewhere
% Options for packages loaded elsewhere
\PassOptionsToPackage{unicode}{hyperref}
\PassOptionsToPackage{hyphens}{url}
\PassOptionsToPackage{dvipsnames,svgnames,x11names}{xcolor}
%
\documentclass[
]{article}
\usepackage{xcolor}
\usepackage{amsmath,amssymb}
\setcounter{secnumdepth}{5}
\usepackage{iftex}
\ifPDFTeX
  \usepackage[T1]{fontenc}
  \usepackage[utf8]{inputenc}
  \usepackage{textcomp} % provide euro and other symbols
\else % if luatex or xetex
  \usepackage{unicode-math} % this also loads fontspec
  \defaultfontfeatures{Scale=MatchLowercase}
  \defaultfontfeatures[\rmfamily]{Ligatures=TeX,Scale=1}
\fi
\usepackage{lmodern}
\ifPDFTeX\else
  % xetex/luatex font selection
\fi
% Use upquote if available, for straight quotes in verbatim environments
\IfFileExists{upquote.sty}{\usepackage{upquote}}{}
\IfFileExists{microtype.sty}{% use microtype if available
  \usepackage[]{microtype}
  \UseMicrotypeSet[protrusion]{basicmath} % disable protrusion for tt fonts
}{}
\makeatletter
\@ifundefined{KOMAClassName}{% if non-KOMA class
  \IfFileExists{parskip.sty}{%
    \usepackage{parskip}
  }{% else
    \setlength{\parindent}{0pt}
    \setlength{\parskip}{6pt plus 2pt minus 1pt}}
}{% if KOMA class
  \KOMAoptions{parskip=half}}
\makeatother
% Make \paragraph and \subparagraph free-standing
\makeatletter
\ifx\paragraph\undefined\else
  \let\oldparagraph\paragraph
  \renewcommand{\paragraph}{
    \@ifstar
      \xxxParagraphStar
      \xxxParagraphNoStar
  }
  \newcommand{\xxxParagraphStar}[1]{\oldparagraph*{#1}\mbox{}}
  \newcommand{\xxxParagraphNoStar}[1]{\oldparagraph{#1}\mbox{}}
\fi
\ifx\subparagraph\undefined\else
  \let\oldsubparagraph\subparagraph
  \renewcommand{\subparagraph}{
    \@ifstar
      \xxxSubParagraphStar
      \xxxSubParagraphNoStar
  }
  \newcommand{\xxxSubParagraphStar}[1]{\oldsubparagraph*{#1}\mbox{}}
  \newcommand{\xxxSubParagraphNoStar}[1]{\oldsubparagraph{#1}\mbox{}}
\fi
\makeatother


\usepackage{longtable,booktabs,array}
\usepackage{calc} % for calculating minipage widths
% Correct order of tables after \paragraph or \subparagraph
\usepackage{etoolbox}
\makeatletter
\patchcmd\longtable{\par}{\if@noskipsec\mbox{}\fi\par}{}{}
\makeatother
% Allow footnotes in longtable head/foot
\IfFileExists{footnotehyper.sty}{\usepackage{footnotehyper}}{\usepackage{footnote}}
\makesavenoteenv{longtable}
\usepackage{graphicx}
\makeatletter
\newsavebox\pandoc@box
\newcommand*\pandocbounded[1]{% scales image to fit in text height/width
  \sbox\pandoc@box{#1}%
  \Gscale@div\@tempa{\textheight}{\dimexpr\ht\pandoc@box+\dp\pandoc@box\relax}%
  \Gscale@div\@tempb{\linewidth}{\wd\pandoc@box}%
  \ifdim\@tempb\p@<\@tempa\p@\let\@tempa\@tempb\fi% select the smaller of both
  \ifdim\@tempa\p@<\p@\scalebox{\@tempa}{\usebox\pandoc@box}%
  \else\usebox{\pandoc@box}%
  \fi%
}
% Set default figure placement to htbp
\def\fps@figure{htbp}
\makeatother





\setlength{\emergencystretch}{3em} % prevent overfull lines

\providecommand{\tightlist}{%
  \setlength{\itemsep}{0pt}\setlength{\parskip}{0pt}}



 


\makeatletter
\@ifpackageloaded{caption}{}{\usepackage{caption}}
\AtBeginDocument{%
\ifdefined\contentsname
  \renewcommand*\contentsname{Table of contents}
\else
  \newcommand\contentsname{Table of contents}
\fi
\ifdefined\listfigurename
  \renewcommand*\listfigurename{List of Figures}
\else
  \newcommand\listfigurename{List of Figures}
\fi
\ifdefined\listtablename
  \renewcommand*\listtablename{List of Tables}
\else
  \newcommand\listtablename{List of Tables}
\fi
\ifdefined\figurename
  \renewcommand*\figurename{Figure}
\else
  \newcommand\figurename{Figure}
\fi
\ifdefined\tablename
  \renewcommand*\tablename{Table}
\else
  \newcommand\tablename{Table}
\fi
}
\@ifpackageloaded{float}{}{\usepackage{float}}
\floatstyle{ruled}
\@ifundefined{c@chapter}{\newfloat{codelisting}{h}{lop}}{\newfloat{codelisting}{h}{lop}[chapter]}
\floatname{codelisting}{Listing}
\newcommand*\listoflistings{\listof{codelisting}{List of Listings}}
\makeatother
\makeatletter
\makeatother
\makeatletter
\@ifpackageloaded{caption}{}{\usepackage{caption}}
\@ifpackageloaded{subcaption}{}{\usepackage{subcaption}}
\makeatother
\usepackage{bookmark}
\IfFileExists{xurl.sty}{\usepackage{xurl}}{} % add URL line breaks if available
\urlstyle{same}
\hypersetup{
  pdftitle={The Secret Map of Nature: How Food Webs Tell the Story of Our Planet},
  pdfauthor={Tanya Strydom},
  pdfkeywords={food webs},
  colorlinks=true,
  linkcolor={blue},
  filecolor={Maroon},
  citecolor={Blue},
  urlcolor={Blue},
  pdfcreator={LaTeX via pandoc}}



\title{The Secret Map of Nature: How Food Webs Tell the Story of Our
Planet}
\author{Tanya Strydom %
%
\textsuperscript{%
%
1%
}%
}
\date{2026-02-13}

\usepackage{setspace}
\usepackage[left]{lineno}
\usepackage[letterpaper]{geometry}

\usepackage[nolists,noheads,markers]{endfloat}
\geometry{margin=2.5cm}

\begin{document}

\thispagestyle{empty}
{\bfseries\sffamily\Large The Secret Map of Nature: How Food Webs Tell
the Story of Our Planet}
\vfil
Tanya Strydom %
%
\textsuperscript{%
%
1%
}%

\vfil
{\small
\textbf{Abstract:} TODO
\vfil
\textbf{Keywords:} %
%
food webs%
}
\clearpage
\setcounter{page}{1}
\doublespacing
\linenumbers


\section{What is a Food Web?}\label{what-is-a-food-web}

Imagine a giant map where every species in a forest or ocean is a dot,
and every time one animal eats another, a line connects them. This map
is called a food web.

Food webs aren't just lists of `who eats who'. They are actually `energy
maps'. Think of energy like a battery: plants get their energy from the
sun, and when a rabbit eats a plant, it's like plugging into that
battery. When a fox eats the rabbit, the energy moves again. By looking
at these links, scientists can see the `bigger picture' of how an entire
environment functions and stays healthy.

\section{The Two Types of `Maps'}\label{the-two-types-of-maps}

In our research, we found that there are actually two different ways to
look at these maps:

\begin{itemize}
\item
  The `Maybe' Map (The Metaweb): This is a list of all the feeding links
  that could happen based on how animals have evolved. For example, a
  lion could eat a zebra because it has the right teeth and speed. This
  map tells us about the `potential' for interactions everywhere.
\item
  The `Actually' Map (The Realised Web): This map shows what is actually
  happening in one specific place at one specific time. Just because a
  lion could eat a zebra doesn't mean it will if there are no zebras
  nearby, or if it finds an easier meal elsewhere.
\end{itemize}

{[}Image comparing a `Potential' web with many lines to a `Realised' web
with fewer, specific lines between local species{]}

Why do we care about these maps?

\section{Predicting the Future}\label{predicting-the-future}

If a new species moves into a forest (an `invasive species'), we can use
our `Maybe Map' to guess who they might eat or who might eat them. This
helps us protect native wildlife before problems even start. It also
helps with conservation: we know that to save a predator like a sea
otter, we also have to protect the species it depends on for food.

\section{The Domino Effect (Propagation of
Change)}\label{the-domino-effect-propagation-of-change}

In a food web, everything is connected. If one species disappears, it's
like pulling a thread in a sweater---the whole thing can start to
unravel.

Secondary Extinctions: If a predator's only food source disappears, the
predator might go extinct too, even if nothing else changed.

Rewiring: Sometimes, animals are smart! If their favourite food
disappears, they might ``rewire'' their behaviour and start eating
something else from their ``Maybe Map''.

\section{The Big Picture: Keeping the
Balance}\label{the-big-picture-keeping-the-balance}

By studying both the `Maybe' and the `Actually' maps, scientists can
understand how nature stays stable. It's a delicate balance:

\begin{itemize}
\item
  Bottom-Up: Having enough plants to provide energy for everyone.
\item
  Top-Down: Having predators to make sure no one group (like deer or
  rabbits) grows too large and eats all the plants.
\end{itemize}

\section{Your Mission: Be a
Web-Watcher!}\label{your-mission-be-a-web-watcher}

Understanding these interactions is the key to protecting our planet.
When you see a bird catching a worm or a bee visiting a flower, you
aren't just seeing a snack---you're seeing a tiny piece of a massive,
global network that keeps our world green and functioning!.

\section*{References}\label{references}
\addcontentsline{toc}{section}{References}

\phantomsection\label{refs}





\end{document}
